% Created 2018-01-31 水 09:40
% Intended LaTeX compiler: pdflatex
\documentclass{article}
\usepackage[utf8]{inputenc}
\usepackage[T1]{fontenc}
\usepackage{textcomp}
\usepackage{marvosym}
\usepackage{amsmath}
\usepackage{amssymb}
\usepackage{hyperref,graphicx}
\usepackage{helvet,times,courier}
\tolerance=1000
\usepackage[pxjahyper]{}
\usepackage{color}
\usepackage[cache=false]{minted}
\usepackage{listings}
\author{Masataro Asai}
\date{\textit{<2018-01-31>} }
\title{Assignment No.1}
\begin{document}

\maketitle

\section{Implementing a WHILE-program compiler in Common Lisp}
\label{sec:orgbfa9313}

In order to verify the resulting program for the assignments, I implemented a
WHILE-program compiler in Common Lisp.  To simplify the implementation, I opted
to avoid the original ABI where the output is stored in , and instead just
use the function return stack. This made the problem of returning multiple
values (in \texttt{binarize}) not possible, though.

Since \texttt{while} requires a non-standard evaluation order, I implemented it as a macro.
\texttt{diff} can be implemented straightforwardly as a function.

\lstset{language=Lisp,label= ,caption= ,captionpos=b,numbers=none}
\begin{lstlisting}
(defpackage while
  (:export :while :if :oddp :binarize :diff :>))
(defpackage while.impl
  (:use :cl :iterate :alexandria :trivia)
  (:shadowing-import-from :while :while))
(in-package :while.impl)

;; blah blah blah.

(defmacro while:while (condition &body body)
  `(do ()
       ((= 0 ,condition))
     ,@body))

(defun diff (a b) (max (- a b) 0))
\end{lstlisting}

\section{\texttt{if}}
\label{sec:orgdd60bb7}

Since \texttt{if} requires a non-standard evaluation order, I implemented it as a macro.

\lstset{language=Lisp,label= ,caption= ,captionpos=b,numbers=none}
\begin{lstlisting}
(defmacro while:if (condition then else)
  (with-gensyms (aux aux2 result)
    `(let ((,aux 1) (,aux2 1) (,result 1))
       (setf ,aux ,condition
             ,aux2 1)
       (while ,aux
         (setf ,result ,then
               ,aux 0
               ,aux2 0))
       (while ,aux2
         (setf ,result ,else
               ,aux2 0))
       ,result)))

(print (while:if 1 1 0)) ; -> 1
(print (while:if 0 1 0)) ; -> 0
\end{lstlisting}

\section{\texttt{>}}
\label{sec:org209c9e2}

Using \texttt{if}, implementing \texttt{>} is trivial.

\lstset{language=Lisp,label= ,caption= ,captionpos=b,numbers=none}
\begin{lstlisting}
(defun while:> (a b)
  (let ((c (diff a b)))
    (while:if c
              1
              0)))

(print (while:> 5 4)) ; -> 1
(print (while:> 4 5)) ; -> 0
\end{lstlisting}

\section{\texttt{oddp}}
\label{sec:org88529c7}

Implemented \texttt{oddp} trivially, which is .

\lstset{language=Lisp,label= ,caption= ,captionpos=b,numbers=none}
\begin{lstlisting}
(defun while:oddp (thing)
  "O(n)"
  (while:if thing
            (let ((aux (diff thing 1)))
              (while:if aux
                        (while:oddp (diff aux 1))
                        1))
            0))

(print (while:oddp 5)) ; -> 1
(print (while:oddp 4)) ; -> 0
\end{lstlisting}

\section{\texttt{binarize}}
\label{sec:orgbd31612}

Due to the problem returning multiple values in a vector, I instead just printed it to the standard output.

The following implementation constructs a decision tree with a decision node
for each assuming the maximum binary digit of the given number
is finite. It thus achieves . Using the similar technique I can also
implement a logarithmic \texttt{oddp}.

\lstset{language=Lisp,label= ,caption= ,captionpos=b,numbers=none}
\begin{lstlisting}
(defmacro while:binarize (thing d)
  (check-type d integer)
  (with-gensyms (aux)
    (once-only (thing)
      (if (= d 0)
          `(while:if ,thing (progn (princ 1) 1) (progn (princ 0) 0))
          `(let ((,aux (diff ,thing (expt 2 ,d))))
             (while:if ,aux
                       (progn (princ 1) (while:binarize ,aux ,(1- d)))
                       (progn (princ 0) (while:binarize ,thing ,(1- d)))))))))

(print (while:binarize 5 4))
; => 00101
; -> 1
\end{lstlisting}

The result of expanding \texttt{(while:binarize 3 2)} is as follows. \texttt{with-gensyms}
creates a new symbol which does not conflict to the existing symbols, similar to
adding a new .

\lstset{language=Lisp,label= ,caption= ,captionpos=b,numbers=none}
\begin{lstlisting}
(LET ((#:THING1251 3))
  (LET ((#:AUX1250 (DIFF #:THING1251 (EXPT 2 2))))
    (WHILE:IF #:AUX1250
              (PROGN (PRINC 1)
                     (LET ((#:THING1253 #:AUX1250))
                       (LET ((#:AUX1252 (DIFF #:THING1253 (EXPT 2 1))))
                         (WHILE:IF #:AUX1252
                                   (PROGN (PRINC 1)
                                          (LET ((#:THING1255 #:AUX1252))
                                            (WHILE:IF #:THING1255
                                                      (PROGN (PRINC 1) 1)
                                                      (PROGN (PRINC 0) 0))))
                                   (PROGN (PRINC 0)
                                          (LET ((#:THING1257 #:THING1253))
                                            (WHILE:IF #:THING1257
                                                      (PROGN (PRINC 1) 1)
                                                      (PROGN (PRINC 0) 0))))))))
              (PROGN (PRINC 0)
                     (LET ((#:THING1259 #:THING1251))
                       (LET ((#:AUX1258 (DIFF #:THING1259 (EXPT 2 1))))
                         (WHILE:IF #:AUX1258
                                   (PROGN (PRINC 1)
                                          (LET ((#:THING1261 #:AUX1258))
                                            (WHILE:IF #:THING1261
                                                      (PROGN (PRINC 1) 1)
                                                      (PROGN (PRINC 0) 0))))
                                   (PROGN (PRINC 0)
                                          (LET ((#:THING1263 #:THING1259))
                                            (WHILE:IF #:THING1263
                                                      (PROGN (PRINC 1) 1)
                                                      (PROGN (PRINC 0) 0)))))))))))
\end{lstlisting}
\end{document}